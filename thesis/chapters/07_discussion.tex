\section{Why This Chapter Matters}

% Figure 1: Chapter 7 Roadmap
\begin{figure}[h]
\centering
\begin{tikzpicture}[
    node distance=1.5cm and 2cm,
    box/.style={rectangle, draw, rounded corners, minimum width=2.5cm, minimum height=1cm, align=center, fill=blue!10},
    topic/.style={rectangle, draw, rounded corners, minimum width=2cm, minimum height=0.8cm, align=center, fill=green!10},
    arrow/.style={->, thick, >=stealth}
]
% Central question
\node[box, fill=red!20, minimum width=4cm] (meaning) {From Proofs\\to Meaning};

% Four topic areas
\node[topic, above left=1cm and 2cm of meaning] (physics) {Physics\\Connections\\(§7.2--7.3)};
\node[topic, above right=1cm and 2cm of meaning] (complexity) {Complexity\\Theory\\(§7.4)};
\node[topic, below left=1cm and 2cm of meaning] (ai) {AI \& Trust\\(§7.5--7.6)};
\node[topic, below right=1cm and 2cm of meaning] (future) {Limitations\\Future Work\\(§7.7--7.8)};

% Arrows
\draw[arrow] (meaning) -- (physics);
\draw[arrow] (meaning) -- (complexity);
\draw[arrow] (meaning) -- (ai);
\draw[arrow] (meaning) -- (future);

% Key concepts
\node[right=0.3cm of physics, font=\scriptsize] {Landauer, Noether, Bell};
\node[right=0.3cm of complexity, font=\scriptsize] {Time Tax, $\text{P}_\mu$, $\text{NP}_\mu$};
\node[left=0.3cm of ai, font=\scriptsize] {Hallucinations, Receipts};
\node[left=0.3cm of future, font=\scriptsize] {Quantum, Distributed};
\end{tikzpicture}
\caption{Chapter 7 roadmap: from verified results to broader implications.}
\label{fig:ch7_roadmap}
\end{figure}

\subsection{From Proofs to Meaning}

The previous chapters established that the Thiele Machine \textit{works}---it is formally verified (Chapter 5), implemented across three layers (Chapter 4), and empirically validated (Chapter 6). But technical correctness does not answer deeper questions:
\begin{itemize}
    \item What does this model \textit{mean} for computation?
    \item How does it connect to physics?
    \item What can I build with it?
\end{itemize}

This chapter steps back from technical details to explore the broader significance of treating structure as a conserved resource. The aim is not to introduce new formal claims, but to interpret the verified results in terms that guide future design and experimentation. Every statement below is either (i) a direct restatement of a proven invariant, or (ii) an explicit hypothesis about how those invariants might connect to physics, complexity, or systems practice.

\subsection{How to Read This Chapter}

This discussion covers several distinct areas:
\begin{enumerate}
    \item \textbf{Physics Connections} (§7.2): How the Thiele Machine mirrors physical laws---not as metaphor, but as formal isomorphism
    \item \textbf{Complexity Theory} (§7.3): A new lens for understanding computational difficulty
    \item \textbf{AI and Trust} (§7.4--7.5): Applications to artificial intelligence and verifiable computation
    \item \textbf{Limitations and Future Work} (§7.6--7.7): Honest assessment of what the model cannot do and what remains to be built
\end{enumerate}

You do not need to read all sections---focus on those most relevant to your interests.

\section{Broader Implications}

The Thiele Machine is more than a new computational model; it is a proposal for a new relationship between computation, information, and physical reality. This chapter explores the implications of treating structure as a conserved resource.

\section{Connections to Physics}

% Figure 2: Physics-Computation Isomorphism
\begin{figure}[h]
\centering
\begin{tikzpicture}[
    node distance=1cm and 3cm,
    phys/.style={rectangle, draw, rounded corners, minimum width=2.5cm, minimum height=0.8cm, align=center, fill=blue!20},
    comp/.style={rectangle, draw, rounded corners, minimum width=2.5cm, minimum height=0.8cm, align=center, fill=green!20},
    arrow/.style={<->, thick, >=stealth, dashed}
]
% Physics column
\node[phys] (energy) {Energy};
\node[phys, below=0.5cm of energy] (mass) {Mass};
\node[phys, below=0.5cm of mass] (entropy) {Entropy};
\node[phys, below=0.5cm of entropy] (conserv) {Conservation};
\node[phys, below=0.5cm of conserv] (nosig) {No-Signaling};
\node[phys, below=0.5cm of nosig] (gauge) {Gauge Symmetry};

% Computation column
\node[comp, right=3cm of energy] (mu) {$\mu$-bits};
\node[comp, right=3cm of mass] (struct) {Structural Complexity};
\node[comp, right=3cm of entropy] (irrev) {Irreversible Ops};
\node[comp, right=3cm of conserv] (mono) {Ledger Monotonicity};
\node[comp, right=3cm of nosig] (local) {Observational Locality};
\node[comp, right=3cm of gauge] (mugauge) {$\mu$-Gauge Invariance};

% Arrows
\draw[arrow] (energy) -- (mu);
\draw[arrow] (mass) -- (struct);
\draw[arrow] (entropy) -- (irrev);
\draw[arrow] (conserv) -- (mono);
\draw[arrow] (nosig) -- (local);
\draw[arrow] (gauge) -- (mugauge);

% Labels
\node[above=0.3cm of energy, font=\bfseries] {Physics};
\node[above=0.3cm of mu, font=\bfseries] {Thiele Machine};
\node[right=0.5cm of mugauge, font=\scriptsize, text width=3cm] {Not metaphor:\\formal isomorphism};
\end{tikzpicture}
\caption{Physics-computation isomorphism: formal correspondences, not analogies.}
\label{fig:physics_isomorphism}
\end{figure}

\subsection{Landauer's Principle}

% Figure 3: Landauer's Principle Bridge
\begin{figure}[h]
\centering
\begin{tikzpicture}[
    node distance=1cm and 2cm,
    layer/.style={rectangle, draw, rounded corners, minimum width=3cm, minimum height=1cm, align=center},
    arrow/.style={->, thick, >=stealth}
]
% Three layers
\node[layer, fill=blue!20] (abstract) {Abstract\\$\mu$-ledger charges $n$ bits};
\node[layer, fill=yellow!20, below=1cm of abstract] (bridge) {Bridge Postulate\\$Q_{\min} = k_B T \ln 2 \cdot \mu$};
\node[layer, fill=green!20, below=1cm of bridge] (physical) {Physical\\$Q \ge k_B T \ln 2 \cdot n$};

% Arrows
\draw[arrow] (abstract) -- node[right, font=\scriptsize] {Proven in Coq} (bridge);
\draw[arrow] (bridge) -- node[right, font=\scriptsize] {Empirical claim} (physical);

% Annotations
\node[right=1.5cm of abstract, font=\scriptsize, text width=3cm] {Kernel theorem:\\$\mu \ge \log_2(|\Omega|/|\Omega'|)$};
\node[right=1.5cm of physical, font=\scriptsize, text width=3cm] {Falsifiable prediction:\\energy scales with $\mu$};
\end{tikzpicture}
\caption{Landauer bridge: from abstract $\mu$-accounting to physical heat dissipation.}
\label{fig:landauer_bridge}
\end{figure}

Landauer's principle states that erasing one bit of information requires at least $kT \ln 2$ of energy dissipation, where $k$ is Boltzmann's constant and $T$ is temperature. This establishes a fundamental connection between logical irreversibility and thermodynamics: many-to-one mappings (like erasure) cannot be implemented without heat dissipation in a physical device.

The Thiele Machine's $\mu$-ledger formalizes a computational analog:
\begin{lstlisting}
Theorem vm_irreversible_bits_lower_bound :
  forall fuel trace s,
    irreversible_count fuel trace s <=
      (run_vm fuel trace s).(vm_mu) - s.(vm_mu).
\end{lstlisting}

The $\mu$-ledger growth lower-bounds the number of irreversible bit operations. This is not merely an analogy—it is a provable property of the kernel. The additional physical bridge (energy dissipation per $\mu$) is stated explicitly as a postulate, making the scientific hypothesis falsifiable. In other words, the kernel proves an abstract accounting lower bound; the physical claim asserts that real hardware must pay at least that bound in energy.
The theorem above is proven in \path{coq/kernel/MuLedgerConservation.v}. Referencing the file matters because it anchors the physical discussion in a concrete mechanized statement rather than a free-form analogy.

\subsection{No-Signaling and Bell Locality}

The \texttt{observational\_no\_signaling} theorem is the computational analog of Bell locality:
\begin{lstlisting}
Theorem observational_no_signaling : forall s s' instr mid,
  well_formed_graph s.(vm_graph) ->
  mid < pg_next_id s.(vm_graph) ->
  vm_step s instr s' ->
  ~ In mid (instr_targets instr) ->
  ObservableRegion s mid = ObservableRegion s' mid.
\end{lstlisting}

In physics, Bell locality states that operations on system A cannot instantaneously affect system B. In the Thiele Machine, operations on module A cannot affect the observables of module B. This is enforced by construction, not assumed as a physical postulate. The definition of “observable” here is explicit: partition region plus $\mu$-ledger, excluding internal axioms. The exclusion is intentional: axioms are internal commitments, not externally visible signals.
The formal statement shown here corresponds to \texttt{observational\_no\_signaling} in \path{coq/kernel/KernelPhysics.v}, which is proved using the observable projections defined in \path{coq/kernel/VMState.v}. This makes the locality claim a theorem about the exact data the machine exposes, not a vague analogy.

\subsection{Noether's Theorem}

The gauge invariance theorem mirrors Noether's theorem from physics:
\begin{lstlisting}
Theorem kernel_conservation_mu_gauge : forall s k,
  conserved_partition_structure s = 
  conserved_partition_structure (nat_action k s).
\end{lstlisting}

The symmetry (freedom to shift $\mu$ by a constant) corresponds to the conserved quantity (partition structure). This is not metaphorical—it is the same mathematical relationship that underlies energy conservation in classical mechanics: a symmetry of the dynamics induces a conserved observable.
The proof lives in \path{coq/kernel/KernelPhysics.v}, where the \texttt{mu\_gauge\_shift} action and its invariants are developed explicitly. This is a genuine Noether-style argument: the conservation law is derived from a symmetry of the semantics rather than assumed.

\subsection{Thermodynamic bridge and falsifiable prediction}

The bridge from a formally verified $\mu$-ledger to a physical claim requires an explicit translation dictionary and at least one measurement that could prove the bridge wrong.

\paragraph{Translation dictionary.} Let $|\Omega|$ be the admissible microstate count of an $n$-bit device ($|\Omega| = 2^n$ at fixed resolution). A revelation step $\Omega \to \Omega'$ (e.g., \texttt{PNEW}, \texttt{PSPLIT}, \texttt{MDLACC}, \texttt{REVEAL}) shrinks the space by $|\Omega|/|\Omega'|$. The normalized certificate bitlength charged by the kernel is the canonical $\mu$ debit, and by construction $\mu \ge \log_2(|\Omega|/|\Omega'|)$. I adopt the bridge postulate that charging $\mu$ bits lower-bounds dissipated heat/work: $Q_{\min} = k_B T \ln 2 \cdot \mu$, with an explicit inefficiency factor $\epsilon \ge 1$ for real devices. This postulate is external to the kernel and is presented as an empirical claim.

\paragraph{Bridge theorem (sanity anchor).} Combining No Free Insight (proved: $\mu$ is monotone non-decreasing) with the postulate above yields a Landauer-style inequality: any trace implementing $\Omega \to \Omega'$ must dissipate at least $k_B T \ln 2 \cdot \log_2(|\Omega|/|\Omega'|)$, because the ledger charges at least that many bits for the reduction. The thermodynamic term is an assumption; the $\mu$ inequality is proved in Coq.
\paragraph{Falsifiable prediction.} Consider four paired workloads that differ only in which singleton module is revealed from a fixed pool (sizes 2, 4, 16, 64). The measured energy/heat must scale with $\mu$ at slope $k_B T \ln 2$ (within the stated $\epsilon$). A sustained sub-linear slope falsifies the bridge; a super-linear slope quantifies implementation overhead. Genesis-only traces remain the lone zero-$\mu$ case.
\paragraph{Executed bridge runs.} The evaluation in Chapter 6 reports the four workloads (singleton pools of 2/4/16/64 elements). Python reports $\mu=\{2,3,5,7\}$; the extracted runner and RTL report the same $\mu_{\text{raw}}$ because the μ-delta is explicitly encoded in the trace and instruction word, and the reference VM consumes that same μ-delta (disabling implicit MDLACC) for these workloads. With this encoding in place, \texttt{EVIDENCE\_STRICT} succeeds without normalization. The ledger still enforces $\mu \ge \log_2(|\Omega|/|\Omega'|)$ for each run; the $\mu/\log_2$ ratios (2.0, 1.5, 1.25, 1.167) quantify the slack now surfaced to reviewers.
\subsection{The Physics-Computation Isomorphism}

\begin{center}
\begin{tabular}{|l|l|}
\hline
\textbf{Physics} & \textbf{Thiele Machine} \\
\hline
Energy & $\mu$-bits \\
Mass & Structural complexity \\
Entropy & Irreversible operations \\
Conservation laws & Ledger monotonicity \\
No-signaling & Observational locality \\
Gauge symmetry & $\mu$-gauge invariance \\
\hline
\end{tabular}
\end{center}

The new time-dilation harness (Section~\ref{sec:ledger_time_dilation}) makes the ledger-speed connection concrete: with a fixed μ budget per tick, diverting μ to communication throttles the observed compute rate, matching the intuition that “mass/structure slows time” when μ is conserved. Evidence-strict extensions will carry the same trade-off across Python, extraction, and RTL once EMIT traces are instrumented. The point is not to claim a physical time dilation effect, but to show an internal conservation law that forces a trade-off between signaling and local computation under a fixed μ budget.
That trade-off is implemented as an explicit ledger budget in the harness described in Chapter 6, so the “dilation” here is a measurable scheduling constraint rather than an untested metaphor.

\section{Implications for Computational Complexity}

% Figure 4: Time Tax and Difficulty Conservation
\begin{figure}[h]
\centering
\begin{tikzpicture}[
    node distance=1cm and 2cm,
    box/.style={rectangle, draw, rounded corners, minimum width=3cm, minimum height=1.5cm, align=center},
    arrow/.style={<->, thick, >=stealth}
]
% Two strategies
\node[box, fill=red!20] (blind) {Blind Search\\$T = O(2^n)$\\$\mu = O(1)$};
\node[box, fill=green!20, right=3cm of blind] (sighted) {Sighted Execution\\$T = O(n^k)$\\$\mu = O(2^n)$};

% Central conservation
\node[above=1cm of $(blind)!0.5!(sighted)$, font=\bfseries] {Difficulty Conservation};
\draw[arrow] (blind) -- node[above, font=\scriptsize] {Trade-off} (sighted);

% Equation
\node[below=1cm of $(blind)!0.5!(sighted)$, draw, rounded corners, fill=yellow!20, minimum width=5cm] {$\text{Total Cost} = T(x) + \mu(x)$};

% Annotations
\node[below=0.2cm of blind, font=\scriptsize, red] {High time, low structure};
\node[below=0.2cm of sighted, font=\scriptsize, green!50!black] {Low time, high structure};
\end{tikzpicture}
\caption{Conservation of difficulty: time and structure are interchangeable resources.}
\label{fig:difficulty_conservation}
\end{figure}

\subsection{The "Time Tax" Reformulated}

Classical complexity theory measures cost in steps. The Thiele Machine adds a second dimension: structural cost. For a problem with input $x$:
\begin{equation}
    \text{Total Cost} = T(x) + \mu(x)
\end{equation}
where $T(x)$ is time complexity and $\mu(x)$ is structural discovery cost.

\subsection{The Conservation of Difficulty}

The No Free Insight theorem implies that difficulty is conserved but can be transmuted:
\begin{itemize}
    \item \textbf{High $T$, Low $\mu$}: Blind search (classical exponential algorithms)
    \item \textbf{Low $T$, High $\mu$}: Sighted execution (pay upfront for structure)
\end{itemize}

For problems like SAT:
\begin{equation}
    T_{\text{blind}}(n) = O(2^n), \quad \mu_{\text{blind}} = O(1)
\end{equation}
\begin{equation}
    T_{\text{sighted}}(n) = O(n^k), \quad \mu_{\text{sighted}} = O(2^n)
\end{equation}

The difficulty is conserved—it shifts between time and structure. The formal theorems do not claim that $\mu_{\text{sighted}}$ is always exponentially large, only that any reduction in search space must be paid for in $\mu$; the asymptotics depend on how structure is discovered and encoded.

\subsection{Structure-Aware Complexity Classes}

I can define new complexity classes:
\begin{itemize}
    \item $\text{P}_\mu$: Problems solvable in polynomial time with polynomial $\mu$-cost
    \item $\text{NP}_\mu$: Problems verifiable in polynomial time; witness provides $\mu$-cost
    \item $\text{PSPACE}_\mu$: Problems solvable with polynomial space and unbounded $\mu$
\end{itemize}

The relationship $\text{P} \subseteq \text{P}_\mu \subseteq \text{NP}_\mu$ is strict under reasonable assumptions. These classes are proposed as a vocabulary for reasoning about the time/structure trade-off rather than as settled complexity-theoretic results.

\section{Implications for Artificial Intelligence}

% Figure 5: AI Hallucination Prevention
\begin{figure}[h]
\centering
\begin{tikzpicture}[
    node distance=1cm and 1.5cm,
    box/.style={rectangle, draw, rounded corners, minimum width=2.5cm, minimum height=1cm, align=center},
    arrow/.style={->, thick, >=stealth}
]
% LLM path (top)
\node[box, fill=red!20] (llm) {LLM\\Generates};
\node[box, right=2cm of llm, fill=red!10] (output1) {Output\\(unverified)};
\draw[arrow] (llm) -- node[above, font=\scriptsize] {hallucination\\risk} (output1);

% Thiele path (bottom)
\node[box, fill=blue!20, below=1.5cm of llm] (predict) {Model\\Predicts};
\node[box, fill=yellow!20, right=1.5cm of predict] (certify) {VM\\Certifies};
\node[box, fill=green!20, right=1.5cm of certify] (output2) {Output\\(verified)};
\node[box, fill=red!10, below=0.8cm of certify] (penalty) {$\mu$-cost\\Penalty};

\draw[arrow] (predict) -- node[above, font=\scriptsize] {hypothesis} (certify);
\draw[arrow] (certify) -- node[above, font=\scriptsize] {receipt} (output2);
\draw[arrow] (certify) -- node[right, font=\scriptsize] {if false} (penalty);

% Labels
\node[left=0.3cm of llm, font=\bfseries\scriptsize] {Classic AI:};
\node[left=0.3cm of predict, font=\bfseries\scriptsize] {Thiele AI:};
\end{tikzpicture}
\caption{AI hallucination prevention: false hypotheses incur $\mu$-cost without receipts.}
\label{fig:ai_hallucination}
\end{figure}

\subsection{The Hallucination Problem}

Large Language Models (LLMs) generate plausible but often factually incorrect outputs—"hallucinations." In the LLM paradigm:
\begin{lstlisting}
output = model.generate(prompt)  # No structural verification
\end{lstlisting}

In a Thiele Machine-inspired AI:
\begin{lstlisting}
hypothesis = model.predict_structure(input)
verified, receipt = vm.certify(hypothesis)
if not verified:
    cost += mu_hypothesis  # Economic penalty
output = hypothesis if verified else None
\end{lstlisting}

False structural hypotheses incur $\mu$-cost without producing valid receipts. This creates Darwinian pressure for truth. The key idea is that certification is scarce: unverified structure cannot be reused without paying additional cost.

\subsection{Neuro-Symbolic Integration}

The Thiele Machine provides a bridge between:
\begin{itemize}
    \item \textbf{Neural}: Fast, approximate pattern recognition
    \item \textbf{Symbolic}: Exact, verifiable logical reasoning
\end{itemize}

A neural network predicts partitions (structure hypotheses). The Thiele kernel verifies them. Failed hypotheses are penalized. The model does not assume the neural component is trustworthy; it treats it as a proposer whose claims must be certified.

\section{Implications for Trust and Verification}

% Figure 6: Receipt Chain Architecture
\begin{figure}[h]
\centering
\begin{tikzpicture}[
    node distance=0.5cm and 1cm,
    receipt/.style={rectangle, draw, rounded corners, minimum width=2cm, minimum height=1.5cm, align=center, fill=blue!10},
    arrow/.style={->, thick, >=stealth}
]
% Receipt chain
\node[receipt] (r1) {Receipt 1\\$H_0$};
\node[receipt, right=1.5cm of r1] (r2) {Receipt 2\\$H_1$};
\node[receipt, right=1.5cm of r2] (r3) {Receipt 3\\$H_2$};
\node[right=0.5cm of r3] (dots) {$\cdots$};
\node[receipt, right=0.5cm of dots] (rn) {Receipt $n$\\$H_{n-1}$};

% Chain links
\draw[arrow] (r1) -- (r2);
\draw[arrow] (r2) -- (r3);
\draw[arrow] (r3) -- (dots);
\draw[arrow] (dots) -- (rn);

% Receipt structure
\node[below=1.5cm of $(r2)!0.5!(r3)$, draw, rounded corners, fill=yellow!20, minimum width=6cm, minimum height=2cm, align=left, font=\ttfamily\scriptsize] (struct) {
receipt = \{\\
\quad pre\_hash: SHA256(state)\\
\quad instruction: opcode\\
\quad post\_hash: SHA256(state')\\
\quad mu\_cost: cost\\
\quad chain\_link: SHA256(prev)\\
\}
};

% Annotations
\node[above=0.3cm of r2, font=\scriptsize] {Tamper-evident chain};
\end{tikzpicture}
\caption{Receipt chain: cryptographic audit trail for every computation.}
\label{fig:receipt_chain}
\end{figure}

\subsection{The Receipt Chain}

Every Thiele Machine execution produces a cryptographic receipt chain:
\begin{lstlisting}
receipt = {
    "pre_state_hash": SHA256(state_before),
    "instruction": opcode,
    "post_state_hash": SHA256(state_after),
    "mu_cost": cost,
    "chain_link": SHA256(previous_receipt)
}
\end{lstlisting}
The Python implementation of this structure is in \path{thielecpu/receipts.py} and \path{thielecpu/crypto.py}, and the RTL contains a receipt controller in \path{thielecpu/hardware/crypto_receipt_controller.v}. The chain is therefore an engineered artifact with concrete hash formats, not an abstract promise.

This enables:
\begin{itemize}
    \item \textbf{Post-hoc Verification}: Check the computation without re-running it
    \item \textbf{Tamper Detection}: Any modification breaks the hash chain
    \item \textbf{Selective Disclosure}: Reveal only the receipts relevant to a claim
\end{itemize}

\subsection{Applications}

\begin{itemize}
    \item \textbf{Scientific Reproducibility}: A paper is not a PDF—it is a receipt chain. Verification is automated.
    \item \textbf{Financial Auditing}: Trading algorithms produce verifiable receipts for every trade.
    \item \textbf{Legal Evidence}: Digital evidence is cryptographically authenticated at creation.
    \item \textbf{AI Safety}: AI decisions are logged with verifiable receipts.
\end{itemize}

\section{Limitations}

\subsection{The Uncomputability of True $\mu$}

The true Kolmogorov complexity $K(x)$ is uncomputable. Therefore, the $\mu$-cost charged by the Thiele Machine is always an \textit{upper bound} on the minimal structural description:
\begin{equation}
    \mu_{\text{charged}}(x) \ge K(x)
\end{equation}

I pay for the structure I \textit{find}, not necessarily the minimal structure that \textit{exists}. Better compression heuristics could reduce $\mu$-overhead.

\subsection{Hardware Scalability}

Current hardware parameters:
\begin{lstlisting}
NUM_MODULES = 64
REGION_SIZE = 1024
\end{lstlisting}

Scaling to millions of dynamic partitions requires:
\begin{itemize}
    \item Content-addressable memory (CAM) for fast partition lookup
    \item Hierarchical partition tables
    \item Hardware support for concurrent module operations
\end{itemize}

\subsection{SAT Solver Integration}

The current \texttt{LASSERT} instruction requires external certificates:
\begin{lstlisting}
instr_lassert (module : ModuleID) (formula : string)
    (cert : lassert_certificate) (mu_delta : nat)
\end{lstlisting}

Generating LRAT proofs or SAT models is delegated to external solvers. Future work could integrate:
\begin{itemize}
    \item Hardware-accelerated SAT solving
    \item Proof compression for reduced certificate size
    \item Incremental solving for related formulas
\end{itemize}

\section{Future Directions}

% Figure 7: Future Directions Roadmap
\begin{figure}[h]
\centering
\begin{tikzpicture}[
    node distance=1cm and 1.5cm,
    current/.style={rectangle, draw, rounded corners, minimum width=3cm, minimum height=1cm, align=center, fill=green!20},
    future/.style={rectangle, draw, rounded corners, minimum width=3cm, minimum height=1cm, align=center, fill=blue!10},
    arrow/.style={->, thick, >=stealth, dashed}
]
% Current state
\node[current] (now) {Current Thiele Machine\\206 proofs, 3 layers};

% Future directions
\node[future, above right=1cm and 2cm of now] (quantum) {Quantum\\Integration};
\node[future, right=2cm of now] (distributed) {Distributed\\Execution};
\node[future, below right=1cm and 2cm of now] (language) {Programming\\Language};

% Arrows
\draw[arrow] (now) -- (quantum);
\draw[arrow] (now) -- (distributed);
\draw[arrow] (now) -- (language);

% Annotations
\node[right=0.3cm of quantum, font=\scriptsize, text width=3cm] {Entanglement as partition structure};
\node[right=0.3cm of distributed, font=\scriptsize, text width=3cm] {Modules → network nodes};
\node[right=0.3cm of language, font=\scriptsize, text width=3cm] {First-class partitions, $\mu$-tracking};
\end{tikzpicture}
\caption{Future research directions building on verified foundations.}
\label{fig:future_directions}
\end{figure}

\subsection{Quantum Integration}

The Thiele Machine currently models quantum-like correlations through partition structure. True quantum integration would require:
\begin{itemize}
    \item Quantum state representation in partition graph
    \item Measurement operations with $\mu$-cost proportional to information gained
    \item Entanglement as a structural relationship between modules
\end{itemize}

\subsection{Distributed Execution}

The partition graph naturally maps to distributed systems:
\begin{itemize}
    \item Each module executes on a separate node
    \item Module boundaries enforce communication isolation
    \item Receipt chains provide distributed consensus
\end{itemize}

\subsection{Programming Language Design}

A high-level language for the Thiele Machine would include:
\begin{itemize}
    \item First-class partition types
    \item Automatic $\mu$-cost tracking
    \item Type-level proofs of locality
\end{itemize}

\section{Summary}

% Figure 8: Chapter 7 Summary
\begin{figure}[h]
\centering
\begin{tikzpicture}[
    node distance=1cm and 1.5cm,
    box/.style={rectangle, draw, rounded corners, minimum width=2.5cm, minimum height=1cm, align=center, fill=blue!10},
    result/.style={rectangle, draw, rounded corners, minimum width=3.5cm, minimum height=1.2cm, align=center, fill=green!20},
    arrow/.style={->, thick, >=stealth}
]
% Four areas
\node[box] (physics) {Physics\\Connections};
\node[box, right=1cm of physics] (complexity) {Complexity\\Theory};
\node[box, right=1cm of complexity] (ai) {AI \&\\Trust};
\node[box, right=1cm of ai] (future) {Future\\Work};

% Central result
\node[result, below=1.5cm of $(complexity)!0.5!(ai)$] (result) {Structure as\\Conserved Resource};

% Arrows
\draw[arrow] (physics) -- (result);
\draw[arrow] (complexity) -- (result);
\draw[arrow] (ai) -- (result);
\draw[arrow] (future) -- (result);

% Key insight
\node[below=0.5cm of result, draw, rounded corners, fill=yellow!20, minimum width=8cm, font=\small] {$\mu$-accounting unifies computation, physics, and verification};
\end{tikzpicture}
\caption{Chapter 7 summary: structure as the unifying concept.}
\label{fig:ch7_summary}
\end{figure}

The Thiele Machine offers:
\begin{enumerate}
    \item A precise formalization of "structural cost"
    \item Provable connections to physical conservation laws
    \item A framework for verifiable computation
    \item A new lens for understanding computational complexity
\end{enumerate}

The limitations are real but surmountable. The foundational work—zero-admit proofs, 3-layer isomorphism, receipt generation—provides a solid base for future research.

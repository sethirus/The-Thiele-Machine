\documentclass[11pt, a4paper]{report}

\usepackage{amsmath, amssymb, amsthm}
\usepackage{geometry}
\usepackage{enumerate}
\usepackage{hyperref}
\usepackage{mathtools}

\geometry{margin=1in}

\title{\textbf{The Thiele Machine}\\ \large A Complete Mathematical Specification\\[6pt] \normalsize Derived from 285 Machine-Checked Coq Proofs}
\author{Formal Specification}
\date{}

% Environments
\theoremstyle{definition}
\newtheorem{definition}{Definition}[chapter]
\newtheorem{axiom}{Axiom}[chapter]
\newtheorem{example}{Example}[chapter]
\newtheorem{remark}{Remark}[chapter]

\theoremstyle{plain}
\newtheorem{theorem}{Theorem}[chapter]
\newtheorem{lemma}[theorem]{Lemma}
\newtheorem{corollary}[theorem]{Corollary}
\newtheorem{proposition}[theorem]{Proposition}

% Macros
\newcommand{\N}{\mathbb{N}}
\newcommand{\Z}{\mathbb{Z}}
\newcommand{\Reals}{\mathbb{R}}
\newcommand{\B}{\mathbb{B}}
\newcommand{\Sspace}{\mathcal{S}}
\newcommand{\Ispace}{\mathcal{I}}
\newcommand{\Gspace}{\mathcal{G}}
\newcommand{\Mspace}{\mathcal{M}}
\newcommand{\Rspace}{\mathcal{R}}
\newcommand{\muled}{\mu_{\text{ledger}}}
\newcommand{\trace}{\tau}
\newcommand{\Tr}{\operatorname{Tr}}

\begin{document}

\maketitle

\begin{abstract}
This document provides a complete, standalone mathematical specification of the Thiele Machine, a formal model of computation in which structural information carries an explicit, conserved cost $\mu$. The document is organized so that the machine can be reconstructed from first principles using only the mathematics herein.

Every logical inference has been verified by one or more of the 285 machine-checked Coq source files comprising the formal verification suite. However, formal verification establishes that \emph{conclusions follow from axioms}---it does not validate that axioms model physical reality. This document is therefore careful to distinguish three epistemological categories: \textbf{(S)}~Structural theorems about the machine's own mathematics, \textbf{(C)}~Conditional derivations where physics conclusions follow from stated axioms that may or may not hold in nature, and \textbf{(R)}~Consistency relations that verify internal compatibility but do not constitute independent predictions.

The specification covers: the state space and operational semantics; the core ``No Free Insight'' theorem; computational universality; gauge symmetry and Noether conservation; conditional derivations connecting $\mu$-accounting to quantum mechanics (Born rule, Tsirelson bound, no-cloning, unitarity, complex amplitudes, Schr\"odinger equation) under stated physical axioms; consistency relations (Planck's constant); emergent spacetime and thermodynamics; the Thiele Manifold and self-reference tower; and falsifiable predictions.
\end{abstract}

\tableofcontents

%=============================================================================
% PART I: FOUNDATIONS
%=============================================================================

\chapter{Introduction}
The Thiele Machine is a computational model designed to make the thermodynamic cost of structure explicit. Unlike a Turing Machine, which treats all state transitions as cost-equivalent, the Thiele Machine assigns a specific cost $\mu$ to every operation that extracts or asserts structural information.

The central claim, proven formally, is: \emph{one cannot gain structural insight about a system without paying $\mu$-cost}. This specification explores how this constraint, combined with explicit physical axioms (superposition, linearity, information conservation), yields structural results that parallel the laws of quantum mechanics, thermodynamics, and emergent spacetime. The epistemological status of each result---whether it is a structural theorem~\textbf{(S)}, a conditional derivation~\textbf{(C)}, or a consistency relation~\textbf{(R)}---is clearly marked throughout.

This specification is organized into eight parts:
\begin{enumerate}
    \item \textbf{Foundations}: State space, instruction set, operational semantics.
    \item \textbf{Core Theorems}: No Free Insight, $\mu$-Chaitin, cost = complexity.
    \item \textbf{Computation}: Universality, strict subsumption, confluence, halting.
    \item \textbf{Symmetry \& Conservation}: $\mu$-conservation, Noether/gauge, receipts.
    \item \textbf{Quantum Mechanics}: Born rule, Tsirelson, no-cloning, unitarity, complex amplitudes, Schr\"odinger, Planck.
    \item \textbf{Emergent Structure}: Spacetime metric, causal cones, information causality, Landauer bridge.
    \item \textbf{Meta-Theory}: Self-reference, Thiele Manifold, Genesis, three-layer isomorphism, Curry--Howard--Thiele.
    \item \textbf{Predictions}: Falsifiable divergences from standard QM.
\end{enumerate}

\section{Epistemological Framework}

Formal verification (Coq) establishes that conclusions follow logically from axioms. It does \textbf{not} establish that axioms model physical reality. This distinction is critical for evaluating the physics claims in Parts~V--VIII.

Every result in this specification falls into one of three categories:

\begin{description}
    \item[\textbf{(S) Structural}] Theorems about the Thiele Machine as a mathematical object. These are unconditionally true given the definitions. \emph{Examples}: $\mu$-conservation, gauge invariance, confluence, halting undecidability, strict Turing subsumption.

    \item[\textbf{(C) Conditional}] Theorems of the form ``if axiom $X$ holds, then physics result $Y$ follows.'' The logical inference is verified; whether axiom $X$ holds in nature is an empirical question. \emph{Examples}: Born rule uniqueness (conditional on linearity), complex necessity (conditional on 2D amplitudes), unitarity (conditional on information conservation).

    \item[\textbf{(R) Consistency}] Algebraic identities or definitional equivalences that verify internal coherence but do not constitute independent predictions. \emph{Examples}: Planck consistency relation, Tsirelson cost definition.
\end{description}

Each major theorem in Parts~V--VIII is annotated with its category. A theorem marked \textbf{(C)} or \textbf{(R)} is not diminished---conditional derivations and consistency checks are valuable scientific tools---but the reader should not mistake them for derivations from first principles without stated assumptions. See Appendix~B for the complete classification table.

\chapter{The State Space}

\section{Primitives}
Let $\N = \{0,1,2,\dots\}$. Let $\B = \{0,1\}$. Let $\Sigma$ be a finite alphabet.

\section{Regions and Modules}
\begin{definition}[Region]
A \textbf{Region} $R$ is a finite subset of $\N$, canonically represented as a sorted list of unique elements: $\mathrm{norm}(L)$ returns the sorted deduplicated form of list $L$.
\end{definition}

\begin{definition}[Module]
A \textbf{Module} $M = (R_M, A_M)$ consists of a normalized region $R_M \subset \N$ and a set of axioms $A_M \subset \Sigma^*$.
\end{definition}

\section{The Partition Graph}
\begin{definition}[Partition Graph]
A partition graph $G = (N_{id}, \mathcal{T})$ where $N_{id} \in \N$ is the next available module ID and $\mathcal{T}: \{0,\dots,N_{id}-1\} \rightharpoonup \Mspace$ is a finite partial function from IDs to modules.
\end{definition}

\begin{axiom}[Well-Formedness]
$G$ is well-formed iff $\mathrm{dom}(\mathcal{T}) \subseteq \{0,\dots,N_{id}-1\}$.
\end{axiom}

\section{Machine State}
\begin{definition}[VM State]
The complete state is a 7-tuple:
\[ S = (G, C, R, \mathit{Mem}, \mathit{PC}, \mu, \mathit{err}) \]
where $G \in \Gspace$ is the partition graph, $C \in \N^3$ comprises the CSRs (certificate, status, error), $R \in \N^{32}$ is the register file, $\mathit{Mem} \in \N^{256}$ is main memory, $\mathit{PC} \in \N$ is the program counter, $\mu \in \N$ is the $\mu$-ledger, and $\mathit{err} \in \B$ is the global error flag.
\end{definition}

\chapter{Instruction Set and Cost Model}

\section{Instruction Categories}
The instruction set $\Ispace$ is partitioned into three categories. Every instruction $i$ carries an explicit cost parameter $\Delta\mu_i \in \N$.

\subsection{Structural Operations}
These modify the partition graph $G$:
\begin{itemize}
    \item $\texttt{PNEW}(R, \Delta\mu)$: Create module with region $R$.
    \item $\texttt{PSPLIT}(m, L, R, \Delta\mu)$: Split module $m$ into disjoint regions.
    \item $\texttt{PMERGE}(m_1, m_2, \Delta\mu)$: Merge two modules.
    \item $\texttt{PDISCOVER}(m, E, \Delta\mu)$: Attach evidence $E$ to module $m$.
\end{itemize}

\subsection{Logical Operations}
These interact with information content:
\begin{itemize}
    \item $\texttt{LASSERT}(m, \phi, \pi, \Delta\mu)$: Assert formula $\phi$ on module $m$ with proof $\pi$.
    \item $\texttt{LJOIN}(c_1, c_2, \Delta\mu)$: Join certificate checksums.
    \item $\texttt{REVEAL}(m, n, \pi, \Delta\mu)$: Reveal $n$ bits of structure from $m$. \emph{Primary source of $\mu$-cost.}
    \item $\texttt{EMIT}(m, p, \Delta\mu)$: Emit payload $p$ from module $m$.
\end{itemize}

\subsection{Computational Operations (Reversible ALU)}
Standard reversible register/memory operations: $\texttt{XFER}$, $\texttt{XOR\_LOAD}$, $\texttt{XOR\_ADD}$, $\texttt{XOR\_SWAP}$, $\texttt{XOR\_RANK}$.

\section{Intrinsic Cost}
The intrinsic cost function $K : \Ispace \times \Sspace \to \N$:
\begin{itemize}
    \item $K(\texttt{REVEAL}, s) = 1$, \quad $K(\texttt{LASSERT}, s) = 1$ (if structure added), \quad $K(\text{other}, s) = 0$.
\end{itemize}

\chapter{Operational Semantics}

\section{Transition Function}
The transition $\delta: \Sspace \times \Ispace \to \Sspace$ is defined by:
\begin{enumerate}
    \item $\mathit{PC}' = \mathit{PC} + 1$.
    \item $\mu' = \mu + \Delta\mu_i$.
    \item $\mathit{err}' = \mathit{err} \lor \mathrm{fail}(S, i)$.
\end{enumerate}

\section{Traces and Execution}
\begin{definition}[Trace]
A trace $\tau = [i_0, \dots, i_k] \in \Ispace^*$ is a finite sequence of instructions.
\end{definition}

\begin{definition}[Execution]
$\mathrm{Run}([], S) = S$; \quad $\mathrm{Run}(i::\tau', S) = \mathrm{Run}(\tau', \delta(S,i))$.
\end{definition}

\section{Selected Instruction Semantics}

\textbf{PNEW}: Normalize region; if already present, idempotent; else add $(R_{\mathrm{norm}}, \emptyset)$ to $G$.

\textbf{REVEAL}: Update CSRs with checksum of proof $\pi$. Cost: $\Delta\mu$ (typically 1).

\textbf{LASSERT}: Verify proof $\pi$ for formula $\phi$ (LRAT for UNSAT, model-check for SAT). If valid, append $\phi$ to $A_m$; if invalid, set $\mathit{err} \gets \mathrm{true}$. Replaces oracles with verifiable proofs.

%=============================================================================
% PART II: CORE THEOREMS
%=============================================================================

\chapter{No Free Insight}

The central theorem: structural insight is never free.

\section{Definitions}

\begin{definition}[Receipt Predicate]
$P: \mathcal{O} \to \B$ is a computable function on observation histories.
\end{definition}

\begin{definition}[Strength Ordering]
$P_1 \le P_2$ iff $\forall o.\; P_1(o)=1 \implies P_2(o)=1$.
$P_1 < P_2$ iff $P_1 \le P_2$ and $\exists o.\; P_2(o)=1 \wedge P_1(o)=0$.
\end{definition}

\begin{definition}[Certification]
Trace $\tau$ \textbf{certifies} $P$ iff execution succeeds ($\mathit{err}=0$), the supra-certificate flag is set, and the output satisfies $P$.
\end{definition}

\begin{definition}[Structure Addition]
$\mathrm{HasStructureAddition}(\tau)$ is true iff $\tau$ contains a $\texttt{REVEAL}$, $\texttt{LASSERT}$, or equivalent operation.
\end{definition}

\section{The Theorem}

\begin{theorem}[No Free Insight {\normalfont (NoFreeInsight\_Theorem.v)}]
\label{thm:nfi}
Let $P_{\mathrm{strong}} < P_{\mathrm{weak}}$. If a trace $\tau$ starting from a clean state $s_0$ certifies $P_{\mathrm{strong}}$, then:
\[ \mathrm{Certified}(\tau, P_{\mathrm{strong}}) \implies \mathrm{HasStructureAddition}(\tau) \]
\end{theorem}

\begin{proof}[Proof Sketch]
Relies on four kernel axioms: (A1) Non-Forgeability of receipts, (A2) Monotonicity of $\mu$, (A3) Locality (disjoint modules are independent), (A4) Underdetermination (partition structure alone cannot distinguish $P_{\mathrm{strong}}$ from $P_{\mathrm{weak}}$). Certifying the stronger predicate requires excluding cases, which by A4 requires new information, which by definition is a structure addition. No axioms, no admits.
\end{proof}

\begin{corollary}[Cost of Insight]
Gaining insight (strengthening the predicate) implies $\Delta\mu > 0$.
\end{corollary}

\chapter{The $\mu$-Chaitin Theorem}

A Chaitin-style incompleteness theorem denominated in $\mu$-currency.

\begin{theorem}[$\mu$-Chaitin {\normalfont (MuChaitinTheory\_Theorem.v)}]
For any formal theory system $T$ satisfying the $\mu$-Chaitin interface, the number of bits $k$ the theory can certify is bounded by:
\[ k \le |T| + c \]
where $|T|$ is the description length of the theory and $c$ is a fixed overhead constant. The proof chain: $k \le \mathrm{payload} \le \mu\text{-info} \le \mathrm{budget} \le |T| + c$.
\end{theorem}

\chapter{Cost Equals Kolmogorov Complexity}

\begin{theorem}[$\mu$-Bits = Prefix-Free Complexity {\normalfont (CostIsComplexity.v)}]
Let $K(\mathrm{spec})$ denote the prefix-free Kolmogorov complexity of specification $\mathrm{spec}$. Then:
\[ \mu(\mathrm{spec}) = K(\mathrm{spec}) \]
Specifically, $\mu(\mathrm{spec}) = |\mathrm{spec}| + 1$ (the terminating bit), which equals the minimum description length over all prefix-free programs producing $\mathrm{spec}$.
\end{theorem}

\begin{proof}
The canonical compiler maps $\mathrm{spec} \mapsto \mathrm{spec} \| [\mathrm{true}]$. Any producing program has the form $p = \mathrm{spec} \| [\mathrm{true}]$ with $|p| = |\mathrm{spec}| + 1$. The compiler achieves the minimum, and $\mu$ counts exactly these bits.
\end{proof}

\chapter{No Free Lunch and No Arbitrage}

\section{No Free Lunch (Ghosts Are Impossible)}

\begin{theorem}[No Free Lunch {\normalfont (NoFreeLunch.v)}]
A ``ghost'' is defined as two distinct propositions $p \neq q$ represented by the same physical state. Ghosts are impossible:
\[ \forall\, p \neq q, \;\nexists\, s \text{ s.t.\ } s \text{ faithfully represents both } p \text{ and } q \]
Information cannot exist without physical distinction.
\end{theorem}

\section{No Arbitrage Implies Thermodynamic Potential}

\begin{theorem}[Potential from No-Arbitrage {\normalfont (NoArbitrage.v)}]
Let $w: \Ispace^* \to \Z$ be a cost function satisfying:
\begin{enumerate}
    \item \textbf{Additivity}: $w([]) = 0$ and $w(\tau_1 \cdot \tau_2) = w(\tau_1) + w(\tau_2)$.
    \item \textbf{No-Arbitrage}: For every closed cycle $\tau$ (returning to the starting state), $w(\tau) \ge 0$.
\end{enumerate}
Then there exists a potential function $\phi: \Sspace \to \Z$ such that:
\[ w(\tau) \ge \phi(\mathrm{apply}(\tau, s)) - \phi(s) \]
for every state $s$ and trace $\tau$.
\end{theorem}

\begin{remark}
This is the Second Law of Thermodynamics: consistent cost accounting implies a state function (entropy) bounding all transitions. The potential $\phi$ is the $\mu$-theoretic analogue of free energy.
\end{remark}

%=============================================================================
% PART III: COMPUTATION
%=============================================================================

\chapter{Computational Universality and Subsumption}

\section{Turing Machine Embedding}

\begin{theorem}[Abstract Simulation {\normalfont (Embedding\_TM.v, TM\_to\_Minsky.v)}]
For every Turing Machine $T$, there exists a Minsky counter machine $M$ and a simulation relation $R$ such that $M$ simulates $T$ step-by-step, encoding the tape into two integers:
\[ \mathrm{Left} = \sum_{i=0}^{h-1} t[h{-}1{-}i]\, 2^i, \qquad \mathrm{Right} = \sum_{i=0}^{n-h-1} t[h{+}i]\, 2^i \]
The Thiele Machine simulates Minsky using its reversible ALU. Hence:
\[ \mathrm{TM} \preceq \mathrm{Minsky} \preceq \mathrm{Thiele} \]
\end{theorem}

\section{Strict Containment}

\begin{theorem}[Turing $\subsetneq$ Thiele {\normalfont (Subsumption.v)}]
Classical Turing computation is \textbf{strictly contained} in sighted Thiele computation:
\[ \exists\, p.\; \mathrm{sighted}(p) \wedge \neg\mathrm{turing}(p) \]
There exist programs that are Thiele-computable (using partition structure and $\mu$-accounting) but are not classical Turing programs. The extra structure is genuine new content.
\end{theorem}

\begin{theorem}[Semantic Strictness {\normalfont (ThieleFoundations.v)}]
There exist Thiele traces with identical Turing shadows but non-isomorphic Thiele-level behavior. The partition and $\mu$ layers carry information not present in the Turing skeleton.
\end{theorem}

\section{Halting Undecidability}
\begin{theorem}[Diagonal Argument {\normalfont (HyperThiele\_Halting.v)}]
No total computable function decides the halting problem for the Thiele Machine.
\end{theorem}

\section{Oracle Cost Lower Bound}
\begin{theorem}[Oracle Impossibility {\normalfont (OracleImpossibility.v)}]
Any oracle correctly resolving $n$ independent halting queries must charge $\Delta\mu \ge n$:
\[ \mathrm{Cost}(n \text{ queries}) \in \Omega(n) \]
A zero-cost oracle is logically inconsistent with $\mu$-conservation.
\end{theorem}

\section{Confluence}
\begin{theorem}[Church--Rosser {\normalfont (Confluence.v)}]
For any state $s$ and two independent certificates $c_1, c_2$ (targeting distinct CNF formulas), applying them in either order yields the same $\mu$-cost:
\[ \mu(s_{12}) = \mu(s_{21}) \]
Independent structural operations commute.
\end{theorem}

\section{Tensoriality}
\begin{theorem}[Module Independence {\normalfont (ThieleUnificationTensor.v)}]
For distinct modules $m_1 \neq m_2$, recording discoveries commutes:
\[ \mathrm{record}(\mathrm{record}(G, m_1, e_1), m_2, e_2) \equiv_{\mathrm{ext}} \mathrm{record}(\mathrm{record}(G, m_2, e_2), m_1, e_1) \]
Local per-module operations are tensorial.
\end{theorem}

\section{Amortized Analysis}
\begin{theorem}[Amortization {\normalfont (AmortizedAnalysis.v)}]
For a batch of $T$ instances with discovery cost $D$ and operational cost $O$ per instance:
\[ \mathrm{total\_cost} = B \cdot D + T \cdot O \ge T \cdot O \]
where $B$ is batch count. As $T \to \infty$, average cost $\to O$ (discovery overhead amortizes away).
\end{theorem}

%=============================================================================
% PART IV: SYMMETRY AND CONSERVATION
%=============================================================================

\chapter{$\mu$-Conservation and Receipt Integrity}

\section{The Conservation Law}

\begin{theorem}[$\mu$-Ledger Conservation {\normalfont (MuLedgerConservation.v)}]
For any trace $\tau = [i_0, \dots, i_k]$:
\[ \mu_{\mathrm{final}} = \mu_{\mathrm{init}} + \sum_{j=0}^{k} \mathrm{cost}(i_j) \]
Every instruction increases $\mu$ by exactly its declared cost. No axioms, no admits.
\end{theorem}

\begin{corollary}[Monotonicity]
$\mu$ never decreases: $\mu(s) \le \mu(\mathrm{Run}(\tau, s))$ for all $\tau$.
\end{corollary}

\begin{corollary}[Irreversibility Bound]
$\mathrm{irreversible\_bits}(\tau) \le \mu_{\mathrm{final}} - \mu_{\mathrm{init}}$.
\end{corollary}

\begin{theorem}[$\mu$ Decomposition {\normalfont (MuLedgerConservation.v)}]
$\mu_{\mathrm{total}} = \mu_{\mathrm{blind}} + \mu_{\mathrm{sighted}}$ for any partition of the cost into blind (reversible) and sighted (structural) components.
\end{theorem}

\section{Receipt Integrity}

\begin{definition}[Receipt]
A receipt $r = (\mathrm{step}, \mathrm{instr}, \mu_{\mathrm{pre}}, \mu_{\mathrm{post}}, h_{\mathrm{pre}}, h_{\mathrm{post}})$ records one transition.
\end{definition}

\begin{theorem}[Receipt Chain Validity {\normalfont (ReceiptIntegrity.v)}]
A valid receipt chain proves $\mu_{\mathrm{final}} = \mu_{\mathrm{init}} + \sum \mathrm{costs}$. Any receipt claiming $\Delta\mu \neq \mathrm{cost}(\mathrm{instr})$ fails validation.
\end{theorem}

\begin{theorem}[Non-Forgeability]
Forged receipts (claiming incorrect $\Delta\mu$) fail validation. Overflow values ($\mu > 2^{31}-1$) are rejected.
\end{theorem}

\chapter{Gauge Symmetry and Noether's Theorem}

\section{The $\Z$-Action (Gauge Group)}

\begin{definition}[Gauge Shift]
For $\delta \in \Z$, the gauge shift $\sigma_\delta: \Sspace \to \Sspace$ maps $S \mapsto S[\mu \gets \mu + \delta]$, preserving all other fields.
\end{definition}

\begin{theorem}[Group Action {\normalfont (KernelNoether.v)}]
The gauge shifts form a $\Z$-action on states:
\begin{enumerate}
    \item \textbf{Identity}: $\sigma_0(S) = S$.
    \item \textbf{Composition}: $\sigma_a(\sigma_b(S)) = \sigma_{a+b}(S)$.
    \item \textbf{Inverse}: $\sigma_{-n}(\sigma_n(S)) = S$ (when $\mu + n \ge 0$).
\end{enumerate}
\end{theorem}

\section{Gauge Invariance}

\begin{definition}[Observable]
$\mathrm{Obs}(S)$ extracts the partition region structure, ignoring $\mu$.
\end{definition}

\begin{theorem}[Gauge Invariance {\normalfont (KernelNoether.v)}]
$\mathrm{Obs}(\sigma_\delta(S)) = \mathrm{Obs}(S)$. Shifting $\mu$ by a constant does not affect observables.
\end{theorem}

\begin{theorem}[Noether's Theorem {\normalfont (KernelNoether.v, RepresentationTheorem.v)}]
\leavevmode
\begin{enumerate}
    \item \textbf{Forward}: States identical except for $\mu$ lie on the same gauge orbit.
    \item \textbf{Backward}: $\mu$-conservation implies gauge invariance of observables.
    \item \textbf{Trace Preservation}: Gauge-equivalent states produce identical observable traces (same labels, same $\mu$-costs) for any finite horizon.
    \item \textbf{Observable Completeness}: Trace-equivalent states are gauge-equivalent.
\end{enumerate}
The $\mu$-ledger is a gauge degree of freedom. Its \emph{absolute} value is unobservable; only \emph{differences} $\Delta\mu$ are physical.
\end{theorem}

\begin{theorem}[$\mu$-Monotonicity (Second Law) {\normalfont (KernelNoether.v)}]
$\mathrm{vm\_step}(S, i, S') \implies \mu(S) \le \mu(S')$. The $\mu$-ledger never decreases under the dynamics.
\end{theorem}

%=============================================================================
% PART V: DERIVATION OF QUANTUM MECHANICS
%=============================================================================

\chapter{Two-Dimensional Amplitude Space}

\begin{remark}[Epistemological Status: \textbf{(C)} Conditional Derivation]
The following derivation assumes that partition states admit \emph{amplitude} representations (superpositions), not merely classical probability distributions. This is stated as Axiom~\ref{ax:superposition}. Without it, classical probability theory ($p \in [0,1]$, one-dimensional, continuous) suffices and the 2D argument does not apply. The passage from a discrete binary partition to the continuum $S^1$ also requires a limiting process not formalized in the Coq suite. The argument does not rule out quaternionic (4D) amplitudes; it establishes 2D as the \emph{minimum} dimension for continuous superposition.
\end{remark}

\begin{axiom}[Superposition Principle]
\label{ax:superposition}
Partition module states admit amplitude representations: a state with $n$ classical configurations is described by $n$ real amplitudes $(a_1, \ldots, a_n)$ satisfying $\sum_i a_i^2 = 1$, where $a_i^2$ gives the probability of configuration~$i$.
\end{axiom}

Given this axiom, binary partition structure forces quantum amplitudes to live in at least two dimensions.

\begin{theorem}[1D Is Insufficient {\normalfont (TwoDimensionalNecessity.v)}]
A one-dimensional normalized amplitude satisfies $x^2 = 1$, hence $x \in \{+1, -1\}$. No intermediate superpositions exist.
\end{theorem}

\begin{theorem}[2D Is Necessary and Sufficient {\normalfont (TwoDimensionalNecessity.v)}]
Binary partition structure requires exactly 2D amplitude space. Two-dimensional amplitudes $(a,b)$ with $a^2 + b^2 = 1$ admit a continuous family via $a = \cos\theta$, $b = \sin\theta$: the unit circle $S^1$.
\end{theorem}

\chapter{Complex Amplitudes from Norm Preservation}

Zero-cost evolution must preserve the norm $a^2 + b^2 = 1$. The group of norm-preserving maps on $S^1$ is $\mathrm{SO}(2) \cong U(1)$, forcing amplitudes to be complex numbers.

\begin{theorem}[Rotation Group {\normalfont (ComplexNecessity.v)}]
2D rotations $R_\theta(a,b) = (a\cos\theta - b\sin\theta,\; a\sin\theta + b\cos\theta)$ satisfy:
\begin{enumerate}
    \item $|R_\theta(a,b)|^2 = a^2 + b^2$ (norm preservation).
    \item $R_0 = \mathrm{id}$ (identity).
    \item $R_{\theta_1} \circ R_{\theta_2} = R_{\theta_1 + \theta_2}$ (composition = addition).
    \item $R_\theta^{-1} = R_{-\theta}$ (invertibility).
\end{enumerate}
\end{theorem}

\begin{theorem}[Complex Necessity {\normalfont (ComplexNecessity.v)}]
Complex multiplication by $e^{i\theta}$ is exactly 2D rotation. The norm-preserving maps on $S^1$ are exactly complex multiplications: $\mathrm{SO}(2) \cong U(1)$. Hence quantum amplitudes \emph{must} be complex numbers.
\end{theorem}

\begin{corollary}[Euler's Formula]
$e^{i\theta_1} \cdot e^{i\theta_2} = e^{i(\theta_1 + \theta_2)}$.
\end{corollary}

\textbf{Derivation chain}: Binary partition $\to$ 2D amplitudes $\to$ normalization ($S^1$) $\to$ zero-cost $\Rightarrow$ norm-preserving $\Rightarrow$ $\mathrm{SO}(2) \cong U(1)$ $\Rightarrow$ complex numbers.

\chapter{The Born Rule}

\begin{definition}[Bloch Sphere Probabilities]
For a state with purity vector $(x,y,z)$:
\[ P(|0\rangle) = \frac{1+z}{2}, \qquad P(|1\rangle) = \frac{1-z}{2} \]
\end{definition}

\begin{definition}[Measurement $\mu$-Cost]
\[ \mathrm{Cost}(x,y,z) = \frac{1 - (x^2 + y^2 + z^2)}{2} \]
This is the linear entropy. For pure states ($r^2=1$), cost $= 0$. For mixed states ($r^2 < 1$), cost $> 0$.
\end{definition}

\begin{theorem}[Uniqueness of the Born Rule {\normalfont (BornRule.v)}]
The Born rule $P(|0\rangle) = (1+z)/2$ is the \textbf{unique} probability assignment satisfying:
\begin{enumerate}
    \item Non-negativity and normalization: $P \ge 0$, $P(|0\rangle) + P(|1\rangle) = 1$.
    \item Eigenstate consistency: $P(0,0,1) = 1$, $P(0,0,-1) = 0$.
    \item Linearity in $z$: $P(x,y,z,0) = az + b$ for some $a,b$.
    \item $\mu$-consistency: expected information gain $=$ $\mu$-cost.
\end{enumerate}
Proof: Boundary conditions force $a = 1/2$, $b = 1/2$. No other solution exists.
\end{theorem}

\begin{theorem}[Born Rule from Tensor Consistency {\normalfont (BornRuleFromSymmetry.v)}]
Let $g: [0,1] \to [0,1]$ be any function satisfying:
\begin{enumerate}
    \item Eigenstate consistency: $g(0) = 0$, $g(1) = 1$.
    \item Normalization: $g(x) + g(1-x) = 1$ for all $x \in [0,1]$.
    \item Tensor product consistency: $g(xy) = g(x) \cdot g(y)$ for $x,y \in [0,1]$.
    \item Range: $0 \le g(x) \le 1$ for $x \in [0,1]$.
\end{enumerate}
Then $g(x) = x$ for all $x \in [0,1]$.
\end{theorem}

\begin{remark}[Epistemological Status: \textbf{(C)} Conditional Derivation---Circularity Resolved]
The original proof (\texttt{BornRule.v}) depends on the linearity axiom (item~3): $P$ is affine in $z$. This assumption is equivalent to the Born rule restated, raising a circularity concern.

The new derivation (\texttt{BornRuleFromSymmetry.v}) replaces linearity with \emph{tensor product consistency}: independent measurements yield independent outcomes, i.e., $g(xy) = g(x)g(y)$. This is a structural property of the machine, not a physics axiom---it follows from module independence proven in \texttt{ThieleUnificationTensor.v}. The proof proceeds: $g = \mathrm{id}$ on all dyadic rationals (by multiplicativity + normalization); then monotonicity + density of dyadics forces $g = \mathrm{id}$ everywhere (via the Archimedean property of $\mathbb{R}$). Zero admits.
\end{remark}

\chapter{Unitarity}

\begin{definition}[Evolution]
An evolution $E$ maps Bloch vectors $(x,y,z) \mapsto (x',y',z')$ with associated $\mu$-cost.
\end{definition}

\begin{theorem}[Unitarity from Zero Cost {\normalfont (Unitarity.v)}]
\leavevmode
\begin{enumerate}
    \item $\mu = 0 \implies r'^2 \ge r^2$ (purity cannot decrease at zero cost).
    \item Non-unitary evolution ($r'^2 < r^2$ for some state) requires $\mu > 0$.
    \item $\mu = 0 \Leftrightarrow$ unitary $\Leftrightarrow$ reversible.
\end{enumerate}
\end{theorem}

\begin{theorem}[Lindblad Requires $\mu$ {\normalfont (Unitarity.v)}]
Lindblad-type dissipation at rate $\gamma > 0$ requires $\mu \ge \gamma$. Decoherence is paid for.
\end{theorem}

\begin{theorem}[CPTP Structure {\normalfont (Unitarity.v)}]
Positivity + trace preservation $\implies$ the evolution is CPTP (completely positive, trace-preserving).
\end{theorem}

\chapter{Purification}

\begin{theorem}[Purification Principle {\normalfont (Purification.v)}]
Every mixed state $(x,y,z)$ with $r^2 = x^2+y^2+z^2 \le 1$ admits eigenvalues $\lambda_1, \lambda_2$ with:
\begin{itemize}
    \item $\lambda_1 + \lambda_2 = 1$, \quad $0 \le \lambda_i \le 1$.
    \item $(\lambda_1 - \lambda_2)^2 = r^2$.
    \item Construction: $\lambda_1 = (1 + r)/2$, $\lambda_2 = (1 - r)/2$.
\end{itemize}
Pure states ($r=1$) need no external reference (deficit $= 0$). The maximally mixed state ($r=0$) has deficit $= 1$.
\end{theorem}

\chapter{No-Cloning}

\begin{theorem}[No-Cloning from $\mu$-Conservation {\normalfont (NoCloning.v)}]
Let $I = x^2 + y^2 + z^2$ be the information content (purity). For a cloning operation with outputs of information $I_1, I_2$ and $\mu$-cost $\mu$:
\[ I_1 + I_2 \le I + \mu \]
\end{theorem}

\begin{theorem}[No-Cloning from $\mu$-Monotonicity {\normalfont (NoCloningFromMuMonotonicity.v)}]
Machine-native no-cloning using nat-valued structural content (MDL complexity). If cloning duplicates $n$ bits of structural content while producing zero new bits: $2n \le n + 0$ implies $n \le 0$. The entire proof is a single \texttt{lia} step (linear integer arithmetic). No Bloch sphere, no Hilbert space, no real analysis. Zero admits.
\end{theorem}

\begin{corollary}[Perfect Cloning Is Impossible at Zero Cost]
$I_1 = I_2 = I$ and $\mu = 0$ implies $2I \le I$, hence $I \le 0$. Only the trivial state can be cloned for free.
\end{corollary}

\begin{corollary}[Cloning Cost]
Perfect cloning requires $\mu \ge I$. For pure states: $\mu \ge 1$.
\end{corollary}

\begin{theorem}[Approximate Cloning {\normalfont (NoCloning.v)}]
For fidelities $f_1, f_2$: $f_1 + f_2 \le 1 + \mu/I$. At zero cost: $f_1 + f_2 \le 1$.
\end{theorem}

\begin{theorem}[No Deletion {\normalfont (NoCloning.v)}]
Perfect deletion also requires $\mu \ge I$ (dual of no-cloning).
\end{theorem}

\chapter{Observation and Collapse}

\section{Observation Irreversibility}

\begin{theorem}[REVEAL Is Irreversible {\normalfont (ObservationIrreversibility.v)}]
For $\mathrm{bits} > 0$: $\mu_{\mathrm{after}} > \mu_{\mathrm{before}}$, and the post-REVEAL state $\neq$ the pre-REVEAL state. Observation prevents recovery of the pre-measurement superposition.
\end{theorem}

\section{Collapse Determination}

\begin{definition}[Partition Entropy]
$H(P) = \log_2(\dim(P))$ where $\dim$ is the partition state dimension.
\end{definition}

\begin{theorem}[Maximum Information Implies Unique State {\normalfont (CollapseDetermination.v)}]
If bits revealed $= H(P_{\mathrm{before}})$ (maximum information), then $\dim(P_{\mathrm{after}}) = 1$. The measurement fully collapses the state to a unique outcome. This is the \textbf{projection postulate}, derived rather than assumed.
\end{theorem}

\chapter{The Tsirelson Bound}

\begin{definition}[Correlation $\mu$-Cost]
$\mu_{\mathrm{corr}}(S) = 0$ if $|S| \le 2\sqrt{2}$; otherwise $\mu_{\mathrm{corr}} = |S| - 2\sqrt{2}$.
\end{definition}

\begin{remark}[Epistemological Status: \textbf{(C)} Conditional Derivation---Circularity Resolved]
The original Coq proof (\texttt{TsirelsonDerivation.v}) encoded $2\sqrt{2}$ in the definition of $\mu_{\mathrm{corr}}$, making its derivation status \textbf{(R)}.

This gap is now closed. \texttt{TsirelsonGeneral.v} derives $S^2 \le 8$ from pure algebra: a sum-of-squares identity proves each CHSH row contributes $\le 2(e_1^2 + e_2^2)$, then Cauchy--Schwarz bounds the sum by $4\sum e_i^2 \le 8$. \texttt{TsirelsonFromAlgebra.v} provides a self-contained algebraic derivation with achievability witness ($e = \pm 1/\sqrt{2}$, giving $S = 2\sqrt{2}$) and a verified rational bound ($\sqrt{8} < 5657/2000$). Zero admits in both files.
\end{remark}

\begin{theorem}[Tsirelson from Zero $\mu$ {\normalfont (TsirelsonDerivation.v)} --- \textbf{(R)}]
Total $\mu = 0$ implies $|S_{\mathrm{CHSH}}| \le 2\sqrt{2} \approx 2.828$.
\end{theorem}

\begin{theorem}[CHSH Separation {\normalfont (Deliverable\_CHSHSeparation.v)}]
Strict numerical separation:
\[ 2 \quad < \quad 2\sqrt{2} \approx 2.828 \quad < \quad \frac{16}{5} = 3.2 \]
\begin{itemize}
    \item \textbf{Classical} (local receipts): $|S| \le 2$ (Bell bound).
    \item \textbf{Quantum} (admissible, $\mu = 0$): $|S| \le 2\sqrt{2}$ (Tsirelson bound).
    \item \textbf{Supra-quantum} ($\mu > 0$ allowed): witnesses achieve $|S| = 3.2$.
\end{itemize}
The Tsirelson bound $2\sqrt{2}$ is the maximum correlation purchasable at zero $\mu$-cost. Exceeding it requires paying for additional structural information.
\end{theorem}

\chapter{The Schr\"odinger Equation}

\begin{theorem}[Emergent Schr\"odinger {\normalfont (EmergentSchrodinger.v)}]
The finite-difference Schr\"odinger equation emerges from the partition update rules. For a two-component amplitude state $(a,b)$ with mass $m$, potential $V$, and time step $\Delta t$:
\begin{align}
    a(t+\Delta t) &= a - \frac{\Delta t}{2m} \nabla^2 b + \Delta t \cdot Vb \\
    b(t+\Delta t) &= b + \frac{\Delta t}{2m} \nabla^2 a - \Delta t \cdot Va
\end{align}
The coupling is \textbf{antisymmetric}: $c_{a,\nabla^2 b} = -c_{b,\nabla^2 a}$ and $c_{a,Vb} = -c_{b,Va}$, which is necessary and sufficient for probability conservation $\partial_t(a^2 + b^2) = 0$.
\end{theorem}

\begin{remark}[Epistemological Status: \textbf{(C)} Conditional Derivation]
This is exactly the finite-difference discretization of $i\hbar \partial_t \psi = -\frac{\hbar^2}{2m}\nabla^2\psi + V\psi$ with $\psi = a + ib$. The Schr\"odinger form emerges \emph{conditionally}: given Axiom~\ref{ax:superposition} (which entails 2D amplitudes) and the requirement that zero-cost evolution preserves probability ($\partial_t(a^2+b^2)=0$), the antisymmetric coupling structure is the unique linear solution. The mass $m$ and potential $V$ are parameters, not derived.
\end{remark}

\chapter{Planck's Constant}

\begin{remark}[Epistemological Status: \textbf{(R)} Consistency Relation]
The following is \textbf{not} a derivation of Planck's constant from first principles. It is a consistency relation that gives $h$ a physical interpretation within the $\mu$-framework. The Coq proof verifies an algebraic identity; the physics is in the definitions.
\end{remark}

\begin{definition}[Physical Constants (Normalized Units)]
$k_B = 1/100$, \; $T = 1$, \; $h = 1$. Landauer energy: $E_L := k_B T \ln 2$.
\end{definition}

\begin{definition}[Computational Time Step]
$\tau_\mu := h / (4 E_L)$, the Margolus--Levitin time at Landauer energy.
\end{definition}

\begin{theorem}[Planck Consistency Relation {\normalfont (PlanckDerivation.v)} --- \textbf{(R)}]
\[ h = 4 \cdot E_L \cdot \tau_\mu \]
\end{theorem}

\begin{remark}[Why This Is Circular]
Substituting $\tau_\mu = h/(4E_L)$ yields $h = 4 E_L \cdot h/(4E_L) = h$. The Coq proof reduces to the \texttt{field} tactic (pure algebra). This does \emph{not} predict~$h$; it defines $\tau_\mu$ in terms of $h$ and verifies consistency.

\textbf{What the relation does provide}: a physical interpretation of Planck's constant as $4\times$ the action (energy $\times$ time) of one $\mu$-bit operation at Landauer cost. If taken as a physical claim, the testable content is the predicted time step $\tau_\mu = h/(4 k_B T \ln 2)$, which at room temperature yields $\tau_\mu \approx 5.7 \times 10^{-14}\,\mathrm{s}$ (femtosecond range, consistent with molecular vibration timescales).

\textbf{What would make this non-circular}: defining $\tau_\mu$ as an independent primitive (measured experimentally) and then \emph{deriving} $h = 4 E_L \tau_\mu$ as a \emph{prediction}. The current formalization does not do this.
\end{remark}

%=============================================================================
% PART VI: EMERGENT STRUCTURE
%=============================================================================

\chapter{Emergent Spacetime}

\section{The $\mu$-Metric}

\begin{definition}[$\mu$-Distance {\normalfont (MuGeometry.v)}]
For two states $S_1, S_2 \in \Sspace$ connected by a trace:
\[ d_\mu(S_1, S_2) = \min_{\tau: S_1 \xrightarrow{\tau} S_2} \sum_{i \in \tau} \mathrm{cost}(i) \]
\end{definition}

\begin{theorem}[Metric Properties {\normalfont (MuGeometry.v)}]
$d_\mu$ satisfies positivity and triangle inequality. For reversible paths it is symmetric (metric); for irreversible paths (monotonic $\mu$) it defines a directed causal order.
\end{theorem}

\section{Causal Cones}

\begin{definition}[Future Light Cone {\normalfont (SpacetimeEmergence.v)}]
$C^+(S) = \{ S' \mid \exists \tau.\; S \xrightarrow{\tau} S' \wedge \mathrm{cost}(\tau) \le \mathrm{Budget} \}$.
\end{definition}

This recovers the causal structure of Special Relativity. The ``speed of light'' $c$ corresponds to the maximum rate of information processing: 1 bit per $\mu$-unit per time step.

\section{No-Signaling}

\begin{theorem}[Observational No-Signaling {\normalfont (SpacetimeEmergence.v)}]
Local operations on module $M_A$ cannot change statistics on disjoint module $M_B$ unless a message (carrying $\mu$-cost) traverses between them.
\end{theorem}

\chapter{Information Causality}

\begin{theorem}[IC Bound {\normalfont (InformationCausality.v)}]
For Alice's inputs $\{x_i\}$ and Bob's guesses $\{y_i\}$ given communication $b$:
\[ \sum_i I(x_i : y_i \mid b) \le \Delta\mu_{\mathrm{channel}} \]
Bob's total information about Alice's data is bounded by the $\mu$-capacity of the channel.
\end{theorem}

\begin{remark}
In the Thiele Machine, Information Causality is a tautology of the cost model: you cannot learn more than you pay for.
\end{remark}

\chapter{Thermodynamic Bridge}

\section{Landauer's Principle}

\begin{theorem}[Landauer Derived {\normalfont (LandauerDerived.v)}]
\leavevmode
\begin{enumerate}
    \item Erasing $\ge 1$ bit is irreversible (fan-in $= 2^k > 1$).
    \item Erasure decreases system entropy ($\Delta S_{\mathrm{sys}} < 0$).
    \item For any physical erasure: environment entropy increase $\ge$ bits erased.
    \item \textbf{Thermodynamic bridge}: energy dissipation $Q \ge k_B T \ln 2 \cdot \mu$.
    \item Erasing $n$ bits costs $\ge n$ Landauer units.
    \item Sequential erasures compose: $\mathrm{bits}(e_1) + \mathrm{bits}(e_2) = \mathrm{in}_1 - \mathrm{out}_2$.
\end{enumerate}
The $\mu$-ledger literally counts irreversible bit erasures.
\end{theorem}

\section{Dissipative Embedding}

\begin{theorem}[Dissipative Model {\normalfont (DissipativeEmbedding.v)}]
Irreversible (dissipative) physics embeds into the Thiele VM with $\Delta\mu \ge 1$ per irreversible step. Reversible physics embeds with $\Delta\mu = 0$.
\end{theorem}

\begin{theorem}[Reversible Physics {\normalfont (PhysicsEmbedding.v, WaveEmbedding.v)}]
Reversible lattice gas and wave models embed with $\mu_{\mathrm{final}} = \mu_{\mathrm{init}}$ (zero cost). Particle count, momentum, wave energy, and wave momentum are conserved by the VM.
\end{theorem}

%=============================================================================
% PART VII: META-THEORY
%=============================================================================

\chapter{Self-Reference and the Thiele Manifold}

\section{Self-Reference Escalation}

\begin{theorem}[G\"odel-Style Escalation {\normalfont (SelfReference.v)}]
Any self-referential system $S$ requires a meta-system $\mathrm{Meta}$ with strictly more dimensions:
\[ \mathrm{self\_ref}(S) \implies \exists\, \mathrm{Meta}: \dim(\mathrm{Meta}) > \dim(S) \]
The meta-system inherits self-reference, creating an infinite escalation.
\end{theorem}

\section{The Thiele Manifold}

\begin{definition}[Thiele Manifold {\normalfont (ThieleManifold.v)}]
An infinite tower of systems $\{L_n\}_{n \in \N}$ with:
\begin{itemize}
    \item $\dim(L_n) = 4 + n$ (level $n$ has dimension $4+n$).
    \item $\dim(L_{n+1}) > \dim(L_n)$ (strict enrichment).
    \item Each level reasons about the level below.
    \item Self-reference at level $n$ is answered by level $n+1$.
\end{itemize}
\end{definition}

\begin{theorem}[Projection to Spacetime {\normalfont (ThieleManifold.v)}]
The projection $\pi_4$ collapses the tower to dimension 4 (spacetime). For $n > 0$:
\begin{itemize}
    \item $\pi_4$ is lossy: $\dim(L_n) > 4$.
    \item $\mu$-cost of projection $= \dim(L_n) - 4 = n > 0$.
\end{itemize}
Spacetime is the ``shadow'' of the full manifold.
\end{theorem}

\section{Genesis: Process--Machine Isomorphism}

\begin{theorem}[Genesis {\normalfont (Genesis.v)}]
There is a definitional isomorphism $\mathrm{Proc} \cong \mathrm{Thiele}$ between coherent processes (step + admissibility proof) and Thiele machines:
\begin{align}
    \mathrm{thiele\_to\_proc}(\mathrm{proc\_to\_thiele}(P)) &= P \\
    \mathrm{proc\_to\_thiele}(\mathrm{thiele\_to\_proc}(T, H)) &= T
\end{align}
Any coherent process \emph{is} a Thiele machine, and vice versa.
\end{theorem}

\chapter{Three-Layer Isomorphism}

\begin{theorem}[Full Isomorphism {\normalfont (FullIsomorphism.v)}]
Three implementations --- Coq specification, Python VM, and Verilog RTL --- are isomorphic: for any trace $\tau$:
\[ \mathrm{decode}(S_{\mathrm{Coq}}(\tau)) = \mathrm{decode}(S_{\mathrm{Python}}(\tau)) = \mathrm{decode}(S_{\mathrm{Verilog}}(\tau)) \]
with $\mu(\mathrm{run}(s, \tau)) = \mu(s) + \sum_i \mathrm{cost}(\tau_i)$ at every layer. Transitivity: $\text{Coq} \cong \text{Python} \cong \text{Verilog} \implies \text{Coq} \cong \text{Verilog}$.
\end{theorem}

\chapter{The Curry--Howard--Thiele Correspondence}

\begin{theorem}[Logic--Computation Isomorphism {\normalfont (LogicIsomorphism.v)}]
The Thiele Machine extends the Curry--Howard correspondence:
\begin{center}
\begin{tabular}{lll}
\textbf{Logic} & \textbf{Computation} & \textbf{Thiele} \\
\hline
Proposition & Type & Partition \\
Proof & Program (term) & Trace \\
Cut elimination & $\beta$-reduction & Execution ($\mathrm{Run}$) \\
Proof equivalence & $=_\beta$ & Execution equivalence \\
\end{tabular}
\end{center}
A valid proof corresponds to a terminating execution. Proof equivalence $\Leftrightarrow$ execution equivalence.
\end{theorem}

\begin{theorem}[Logic Is Physics {\normalfont (LogicToPhysics.v)}]
Cut elimination in logic = relational composition in physics:
\[ \mathrm{interp}(\mathrm{cut}(\pi_1, \pi_2)) = \mathrm{rel\_comp}(\mathrm{interp}(\pi_1), \mathrm{interp}(\pi_2)) \]
Logical proof composition maps to physical relation composition. This is the formal nucleus of the claim that logic and physics share the same categorical structure.
\end{theorem}

\chapter{Categorical Structure}

\begin{theorem}[Conservation as Functor {\normalfont (Universe.v)}]
Define two categories:
\begin{itemize}
    \item $\mathbf{C}_{\mathrm{phys}}$: Objects $=$ universe states (particle momenta lists). Morphisms $=$ paths of momentum-conserving interactions.
    \item $\mathbf{C}_{\mathrm{logic}}$: Objects $=$ total momentum values. Morphisms $=$ equality proofs.
\end{itemize}
The functor $F: \mathbf{C}_{\mathrm{phys}} \to \mathbf{C}_{\mathrm{logic}}$ maps $F(s) = \sum s$ (total momentum). Then:
\[ \mathrm{Path}(s_1, s_2) \implies \sum s_1 = \sum s_2 \]
Conservation of momentum emerges as the functorial image. Observation is a structure-preserving map from physics to logic.
\end{theorem}

\chapter{Audit Infrastructure}

\begin{theorem}[CatNet Integrity {\normalfont (CatNet.v)}]
Adding a new entry to a valid hash-chain audit log produces a valid chain. If the logic oracle detects inconsistency, execution halts and no further entries are written (paradox halting).
\end{theorem}

%=============================================================================
% PART VIII: PREDICTIONS
%=============================================================================

\chapter{Falsifiable Predictions}

\section{Linear Scaling of Structural Cost}

\begin{axiom}[Linear Scaling {\normalfont (FalsifiablePrediction.v)}]
The $\mu$-cost of maintaining coherence of $N$ entangled qubits scales as $O(N)$ per time step.
\end{axiom}

\textbf{Prediction}: Large-scale quantum computers will encounter a fundamental (not merely technical) decoherence noise floor proportional to entanglement complexity.

\section{CHSH Regime Separation}

\textbf{Prediction}: Experiments probing the boundary between quantum ($|S| \le 2\sqrt{2}$) and supra-quantum ($|S| > 2\sqrt{2}$) correlations should find that exceeding the Tsirelson bound requires measurably higher energy dissipation, scaling as:
\[ \Delta E \ge k_B T \ln 2 \cdot (|S| - 2\sqrt{2}) \]

\section{Metric Deformation}

Since $d_\mu$ depends on information content, regions of high structural complexity effectively expand the metric.

\textbf{Prediction}: In high-complexity computations, effective signal latency will increase relative to vacuum speed of light.

\section{Architectural Permanence}

\begin{theorem}[Optimal Quartet {\normalfont (ArchTheorem.v)}]
The four partition discovery strategies (Louvain, Spectral, Degree, Balanced) achieve classification accuracy $> 90\%$ across all problem classes. No alternative configuration exceeds this quartet's accuracy. The configuration is architecturally final.
\end{theorem}

\section{Coupling Constant Prediction}

\begin{definition}[Thiele $\alpha$ Limit {\normalfont (PhysicalConstants.v)}]
The asymptotic density of self-referential programs in the $n$-bit state space:
\[ \alpha = \lim_{n \to \infty} \frac{A_{\mathrm{interaction}}(n)}{V_{\mathrm{spacetime}}(n)} = \lim_{n \to \infty} \frac{n+1}{2^n} \]
This converges to 0, but the \emph{non-asymptotic} structure at physical scales should relate to coupling constants.
\end{definition}

%=============================================================================
% APPENDIX
%=============================================================================

\appendix
\chapter{Proof File Index}

The following table maps each theorem in this specification to its Coq source file.

\begin{center}
\small
\begin{tabular}{ll}
\textbf{Result} & \textbf{Coq Source} \\
\hline
No Free Insight (Thm.~\ref{thm:nfi}) & \texttt{nofi/NoFreeInsight\_Theorem.v} \\
$\mu$-Chaitin & \texttt{nofi/MuChaitinTheory\_Theorem.v} \\
Cost = Complexity & \texttt{theory/CostIsComplexity.v} \\
No Free Lunch & \texttt{theory/NoFreeLunch.v} \\
No Arbitrage & \texttt{kernel/NoArbitrage.v} \\
TM Embedding & \texttt{thielemachine/coqproofs/Embedding\_TM.v} \\
Strict Subsumption & \texttt{thielemachine/coqproofs/Subsumption.v} \\
Confluence & \texttt{thielemachine/coqproofs/Confluence.v} \\
Tensoriality & \texttt{thielemachine/coqproofs/ThieleUnificationTensor.v} \\
$\mu$-Conservation & \texttt{kernel/MuLedgerConservation.v} \\
Receipt Integrity & \texttt{kernel/ReceiptIntegrity.v} \\
Noether / Gauge & \texttt{kernel/KernelNoether.v} \\
Gauge Trace Preservation & \texttt{thielemachine/coqproofs/RepresentationTheorem.v} \\
2D Necessity & \texttt{quantum\_derivation/TwoDimensionalNecessity.v} \\
Complex Necessity & \texttt{quantum\_derivation/ComplexNecessity.v} \\
Born Rule & \texttt{kernel/BornRule.v} \\
Born Rule (Tensor) & \texttt{kernel/BornRuleFromSymmetry.v} \\
Unitarity & \texttt{kernel/Unitarity.v} \\
Purification & \texttt{kernel/Purification.v} \\
No-Cloning & \texttt{kernel/NoCloning.v} \\
No-Cloning (Machine-Native) & \texttt{kernel/NoCloningFromMuMonotonicity.v} \\
Observation Irreversibility & \texttt{quantum\_derivation/ObservationIrreversibility.v} \\
Collapse Determination & \texttt{quantum\_derivation/CollapseDetermination.v} \\
Tsirelson Derivation & \texttt{kernel/TsirelsonDerivation.v} \\
Tsirelson (Algebraic) & \texttt{kernel/TsirelsonFromAlgebra.v} \\
CHSH Separation & \texttt{thielemachine/verification/Deliverable\_CHSHSeparation.v} \\
Emergent Schr\"odinger & \texttt{physics\_exploration/EmergentSchrodinger.v} \\
Planck Consistency & \texttt{physics\_exploration/PlanckDerivation.v} \\
Spacetime Emergence & \texttt{kernel/SpacetimeEmergence.v} \\
$\mu$-Geometry & \texttt{kernel/MuGeometry.v} \\
Information Causality & \texttt{kernel/InformationCausality.v} \\
Landauer & \texttt{thermodynamic/LandauerDerived.v} \\
Dissipative Embedding & \texttt{thielemachine/coqproofs/DissipativeEmbedding.v} \\
Physics Embedding & \texttt{thielemachine/coqproofs/PhysicsEmbedding.v} \\
Self-Reference & \texttt{self\_reference/SelfReference.v} \\
Thiele Manifold & \texttt{thiele\_manifold/ThieleManifold.v} \\
Genesis & \texttt{theory/Genesis.v} \\
Full Isomorphism & \texttt{thielemachine/verification/FullIsomorphism.v} \\
Curry--Howard--Thiele & \texttt{thielemachine/coqproofs/LogicIsomorphism.v} \\
Logic Is Physics & \texttt{theory/LogicToPhysics.v} \\
Functor Soundness & \texttt{isomorphism/coqproofs/Universe.v} \\
CatNet Integrity & \texttt{catnet/coqproofs/CatNet.v} \\
Amortization & \texttt{thielemachine/coqproofs/AmortizedAnalysis.v} \\
Optimal Quartet & \texttt{theory/ArchTheorem.v} \\
Halting & \texttt{thielemachine/coqproofs/HyperThiele\_Halting.v} \\
Oracle Impossibility & \texttt{thielemachine/coqproofs/OracleImpossibility.v} \\
\end{tabular}
\end{center}

\chapter{Epistemological Classification}

The following table classifies every major result by its epistemological status.

\begin{center}
\small
\begin{tabular}{lll}
\textbf{Result} & \textbf{Status} & \textbf{Key Assumption} \\
\hline
\multicolumn{3}{l}{\emph{Unconditional structural theorems about the machine}} \\
\hline
No Free Insight & \textbf{(S)} & Module type contract \\
$\mu$-Chaitin & \textbf{(S)} & Module type contract \\
Cost = Complexity & \textbf{(S)} & Prefix-free coding \\
No Free Lunch & \textbf{(S)} & Faithful representation \\
No Arbitrage $\Rightarrow$ Potential & \textbf{(S)} & Additivity + no-arbitrage \\
TM Embedding & \textbf{(S)} & Standard simulation \\
Strict Subsumption & \textbf{(S)} & Partition structure \\
Halting Undecidability & \textbf{(S)} & Diagonalization \\
Oracle Cost & \textbf{(S)} & $\mu$-conservation \\
Confluence & \textbf{(S)} & Module independence \\
$\mu$-Conservation & \textbf{(S)} & Machine semantics only \\
Receipt Integrity & \textbf{(S)} & Hash-chain model \\
Gauge / Noether & \textbf{(S)} & Machine semantics only \\
Genesis & \textbf{(S)} & Definitional isomorphism \\
Three-Layer Isomorphism & \textbf{(S)} & Decode function \\
Curry--Howard--Thiele & \textbf{(S)} & Interpretation map \\
\hline
\multicolumn{3}{l}{\emph{Conditional derivations (valid given stated axioms)}} \\
\hline
2D Necessity & \textbf{(C)} & Superposition axiom (Axiom~\ref{ax:superposition}) \\
Complex Necessity & \textbf{(C)} & 2D + norm preservation \\
Born Rule Uniqueness & \textbf{(C)} & Tensor product consistency (\texttt{BornRuleFromSymmetry.v}) \\
Unitarity from $\mu=0$ & \textbf{(C)} & Info conservation \\
No-Cloning & \textbf{(C)} & Purity = information \\
Purification & \textbf{(C)} & Bloch sphere model \\
Observation Irreversibility & \textbf{(C)} & REVEAL semantics \\
Collapse Determination & \textbf{(C)} & Entropy = $\log \dim$ \\
Emergent Schr\"odinger & \textbf{(C)} & 2D + antisymmetry \\
Landauer & \textbf{(C)} & Fan-in irreversibility \\
Dissipative Embedding & \textbf{(C)} & Irreversibility model \\
Self-Reference & \textbf{(C)} & Dimension = complexity \\
CHSH Separation & \textbf{(C)} & Numerical witnesses \\
\hline
\multicolumn{3}{l}{\emph{Consistency relations (definitions verified, not predictions)}} \\
\hline
Tsirelson Bound & \textbf{(C)} & Algebraic derivation (\texttt{TsirelsonFromAlgebra.v}) \\
Planck's Constant & \textbf{(R)} & $\tau_\mu := h/(4E_L)$ \\
Information Causality & \textbf{(R)} & Tautology of cost model \\
\end{tabular}
\end{center}

\medskip
\noindent\textbf{Legend}: \textbf{(S)} = unconditional theorem about the machine; \textbf{(C)} = valid derivation conditional on stated axiom; \textbf{(R)} = internal consistency check.

\end{document}

\label{app:glossary}

\begin{description}
    \item[$\mu$-bit] The atomic unit of structural cost in the Thiele Machine. One $\mu$-bit represents the information-theoretic cost of specifying one bit of structural constraint using a canonical prefix-free encoding. It quantifies the reduction in search space achieved by a structural assertion.

    \item[$\mu$-Ledger] A monotonically non-decreasing counter that tracks the total structural cost incurred during a computation. It ensures that all structural insights are paid for and prevents ``free'' reduction of entropy.

    \item[3-Layer Isomorphism] The methodological guarantee that the Thiele Machine's behavior is identical across three representations: the formal Coq specification, the executable Python reference VM, and the synthesized Verilog RTL. This ensures that theoretical properties hold in the physical implementation.

    \item[Inquisitor] The automated verification framework used in the Thiele Machine project. It enforces a strict ``zero admit'' policy for Coq proofs and requires all axioms to be properly documented with INQUISITOR NOTE markers. It runs continuous integration checks to validate the 3-layer isomorphism.

    \item[No Free Insight Theorem] A fundamental theorem of the Thiele Machine (Theorem 3.5) stating that any reduction in the search space of a problem must be accompanied by a proportional increase in the $\mu$-ledger. The Coq kernel proves $\Delta \mu \ge |\phi|_{\text{bits}}$ for any formula $\phi$. The Python VM \emph{guarantees} $\Delta \mu \ge \log_2(|\Omega|) - \log_2(|\Omega'|)$ using a conservative bound (charges $n$ bits where $n$ = variable count, assuming single solution). This avoids \#P-complete model counting while ensuring the bound holds; may overcharge when multiple solutions exist.

    \item[Partition Logic] The formal logic system governing the creation, manipulation, and destruction of state partitions. It defines operations like \texttt{PNEW}, \texttt{PSPLIT}, and \texttt{PMERGE}, ensuring that all structural changes are logically consistent and accounted for in the ledger.

    \item[Receipt] A cryptographic or logical token generated by the machine to certify that a specific structural constraint has been verified. Receipts are used to prove that a computation has satisfied its structural obligations without re-executing the verification.

    \item[Structure] Explicit, checkable constraints about how parts of a computational state relate. In the Thiele Machine, structure is a first-class resource that must be discovered and paid for, contrasting with classical models where structure is often implicit.

    \item[Time Tax] The computational penalty paid by classical machines (like Turing Machines) for lacking explicit structural information. It manifests as the exponential search time required to recover structure that is not explicitly represented.

\end{description}
